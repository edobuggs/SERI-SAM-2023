\documentclass[10pt]{article}
\usepackage{srcltx}
%\usepackage[letterpaper,urlcolor=blue,dvips,colorlinks=true]{hyperref}

\textwidth 6 in \textheight 8.75 in

\voffset-.75in \oddsidemargin.25in \evensidemargin.25in

\begin{document}

\begin{center}
 {\huge \bf Riverside Research Institute}
\end{center}

{\bf To:} Biomed group

{\bf From:} Jeff Ketterling

{\bf Date:} Updated \today

{\bf Subject:} LabVIEW programs as of now
 \vspace{0.5 cm}

The LabVIEW programs written so far are briefly discussed,
including the subprograms. These programs evolve quickly and may
not reflect the most current version. Most programs have notes
included in the diagram window of the VI. Theses notes indicate
what the programs are doing. Some are more straightforward than
others. The programs will be listed as I now have them divided
into folders. Some calls to subVIs are made between the divisions
I have set up. The programs listed under the Main programs
headings have a user friendly interface and are meant for a
specific purpose. The Subprograms sections list the support
programs used by the Main programs.

\vspace{0.5 cm}
 \hline
\vspace{0.5 cm}

\section{ViewData}
\subsection{Main programs}
\subsubsection{{\it AlterHeader.vi}}

Allows header info in EYE files to be altered on a field by field
basis. Can change any number of selected EYE files in a single
directory.

\subsubsection{{\it BinFFT.vi}}

This is a sub-program used by SpecFilter.vi. It is where most of
the processing occurs.

\subsubsection{{\it BytetoBMP.vi}}

Convert an 8 bit byte stream file into a BMP image. Used for
images created for prostate work but can be adapted for other
uses.

\subsubsection{{\it InterpSecImage.vi}}

This program is now obsolete. This VI converts a sector scan into
a polar type of display. The implementation is rather slow because
the intensity values are interpolated to give a final image evenly
spaced in $x$ and $y$ coordinates.

\subsubsection{{\it SpecFilter.vi}}

This program is a method of reducing speckle when creating B-mode
images from EYE files. Take an FFT of each scan line of the eye
file, separate the transmission band into several bins, convert
back to time domain, add the bins, and then display the image.

\subsubsection{{\it ViewEye.vi}}

Multipurpose program to view an eye file and blow up certain
features. A section of the original data can be boxed to give a
3-D rotational view of the data. The data can also be viewed as
individual lines. If the image is a sector scan, a conversion is
made to display it as a true sector scan image. Can also select to
see a pop up window with all the parameters in the EYE file
header.

\subsubsection{{\it ViewPicts.vi}}

Little program to view a sequence of images as a series of
bitmaps. If the images are sector scans, they are displayed
accordingly. The name of the file is overlaid on the image in the
upper right hand corner. User chooses a directory and then the EYE
files to view as Picts.



\subsection{Subprograms}

\subsubsection{{\it ReadEye.llb}}

This library contains VIs to extract info from the EYE files.
Right now they can also read Ron's Cornell linear scan format to a
limited extent. The following VI's extract info from the header as
it is lumped in our EYE format: {\it A/DInfo.vi,
CalibrationInfo.vi, HeaderInfo.vi, OtherInfo.vi, PatientInfo.vi,
RFInfo.vi, ScanInfo.vi, ScanSystemInfo.vi, TGC\&RF\_ADInfo.vi,
TherapyInfo.vi, Time/DateInfo.vi, TissueType.vi,
TransducerInfo.vi}. There are also some more specialized VIs.

\begin{enumerate}
\item {\it MakeHeader.vi}

Assemble all header info into a string format which can be
appended to an EYE file.

\item {\it ReadHead.vi}

Read all the info from the header and makes it available in
clusters corresponding to the info sections just listed above.
Also output the whole 1024 byte header.

\item {\it ReadEyeFile.vi}

Reads info from the EYE file and makes it available as a 2-D array
with the columns corresponding to scan lines. Also makes some
other variable available.

\item {\it RonLinear.vi}

Like {\it ReadEyeFile.vi} but for Cornell linear scan format.

\item {\it SetStringLength.vi}

Sets the length of a string to equal a specified number of
characters. Used to make the text fields the appropriate length in
the header.

\item {\it ViewLine.vi}

Makes a single line from an EYE file available for viewing.

\end{enumerate}

\subsubsection{{\it SectorDisplay.llb}}

This library contains the VIs used to convert a rectangular array
of data into a sector image.

\begin{enumerate}
\item {\it AxisRanges.vi}

Pull out the max and min x and y values of the sector image once
it has been converted from polar to rectangular. Used to scale the
data and represent distance on the x and y axis.

\item {\it BytestoArray.vi}

Convert 8 bit values to scaled values. Need to know high and low
scaling parameters used to create byte file.

\item {\it CheckForSec.vi}

Looks at file header to see if EYE file is a sector scan. If so
outputs a boolean argument. One output for if it is a sector scan
and another for if it is a scan greater than 180 degrees.

\item {\it ConvertToSec.vi}

Converts an image to a sector format. Outputs the z values as an
array and a bitmap.

\item {\it InterpAxis.vi}

Used when forming a new sector image by interpolating along the x
and y axis. Called by {\it InterpSecImage.vi} and therefore not
used anymore.

\item {\it InterpHoles.vi}

Take a freshly mapped sector image and fill in ``holes'' where no
value was inserted. Fills in a hole with an interpolated value
along the horizontal axis (cross range direction).

\item {\it LogScaleImage.vi}

Takes the absolute value of the image data and compresses it
logarithmically. Then scales the data so a value of 128 is still
128 after scaling.

\item {\it MakePict.vi}

Convert a data array to a Pict format. Have choice of how to scale
data. Can use logarithmic compression of linear scaling. All
methods scale back to 256.

\item {\it OverlaySectorRoi.vi}

Overlays a rectangular ROI onto the sector image. The reverse is
not set up. That is, an ROI cannot be chosen on the sector image
which then maps back to a rectangle on the original non-sector
image.

\item {\it OverSample.vi}

Use this for Hitachi data when converting to sector format. The RF
data is first reduced down and then each scan line is duplicated
twice before converting to a sector format. This fills out the
image at the far ranges and leads to a better looking image.

\item {\it PictFromEye.vi}

Make a Pict of the EYE file data and overlay the name of the EYE
file on the image.

\item {\it PolarToRect.vi}

Convert the original polar coordinates of the scan data into
rectangular ($X,Y$) coordinates. The RF data can also be reduced
to speed up image conversion. This is done by reducing the rows
(RF pts) by some user selected value.

\item {\it RescaleAxis.vi}

Re-scale the ($X,Y$) data that has been converted from polar
coordinates so that the max $X$ and $Y$ values correspond to the
number of rows and columns, respectively, in the final sector
image. The ($X,Y$) data now spans from zero to the number of rows
or columns and the data can be dropped into an empty matrix with
the indices corresponding to the re-scaled ($X,Y$) values. Note
($X,Y$) gets rounded off when dropped into a display matrix.

\item {\it ROIEdges.vi}

Determine the coordinates of the rectangular edges of the ROI.
Used to overlay the corresponding ROI on the sector image.

\item {\it ScaleAxisBack.vi}

Take indices in sector image and convert them back to $(X,Y)$
coordinates or to indices from original data array.

\item {\it SectorOverlay.vi}

This is where the sector image is inserted into an empty array.
Uses pixel locations to replace empty elements with image values.
Image has ``holes'' filled in with interpolation by InterpHoles.vi
and is then smoothed with {\it SmoothImage.vi}.

\item {\it SmoothImage.vi}

Rough smoothing of sector image. Can either take an average of the
surrounding points or take the max value of the surrounding
points.

\end{enumerate}

\section{FreqAnalysis}


\subsection{Main programs}

\subsubsection{{\it 1-DSpecViewAndFit.vi}}

Allows a spectrum to be taken for the selected ROI. The resulting
spectrum can be used to make a CAL file or to find fit parameters
for a specified frequency range. The CAL file and fit range can be
read from the file header or input by the user. The resulting fit
parameters can be saved to a {\it Spectra.txt} file with the
format currently in use. The user can specify if the data should
be forced to a certain length FFT. This may be important if a CAL
is subtracted from the data. The ROI needs to be twice the length
of the cal file.

\subsubsection{{\it 2-DSpecView.vi}}

Will perform  2-D spectra analysis on the selected ROI. Can
subtract a CAL from the data if desired.


\subsubsection{{\it AOIParams.vi}}

Allows sliding window data taken with {\it WholeImageROIFit.vi} to
be analyzed with a cylindrical Area of Interest (AOI) or a free
drawn AOI. The parameter values (midband fit, slope, and
intercept) in the AOI are averaged to give a single fit value for
the AOI. Can look at any of the fit images and the results come
out for all three.

\subsubsection{{\it BiopsyBatch.vi}}

Process prostate EYE file data from B\&K machines either on a case
by case basis or as a batch. Biopsy AOI is taken based on a
straight line in the sector co-ordinate system which is then
converted back to the raw data form. AOI can either be between two
parallel lines or a fixed number of points (normally 110). Batch
processing is either done with all EYE files in a single directory
or from a text list of files. Data is saved in the Spectra.txt
format.

\subsubsection{{\it Color.vi}}

Open parameter image files and lookup table, then color code
image.

\subsubsection{{\it ConAgent.vi}}

Process contrast agent data.

\subsubsection{{\it ScatProps.vi}}

Compute $CQ^{2}$ properties. Program is rough and not ready for
any particular use.

\subsubsection{{\it SectorBiopsy.vi}}

Program is similar to BiopsyBatch.vi but is geared more towards
data from the Hitachi. A rectangular ROI is define in the sector
image and converted to the raw data format. The region of the ROI
is then extracted from the data as a sub-set and sliding FFT
analysis is performed on it, creating a parameter image of the
ROI. An average value is then found for the ROI. The Hitachi cross
calibration file is used for the analysis. Program can be run on a
case by case basis or in batch mode. Results are saved in
Spectra.txt.

\subsubsection{{\it WholeImageROIFit.vi}}

Analyze an EYE file with a sliding ROI and return the resulting
slope, intercept, and mid-band fit data. A fit is also made to the
Cal data and this can be subtracted from the final results. The
program  is presently geared to make parameter images that cover
all the scan lines with a 64 pt window that increments by 8 RF
points. The program can be easily modified, though, to allow the
size or the ROI and how may points should be in the $X$ and $Y$
direction of the final image. Data can be saved in a variety of
formats and is put in a folder named Param Images in the same
directory as the EYE files.


\subsection{Subprograms}

\subsubsection{{\it FreqAnalysis.llb}}


\begin{enumerate}
\item {\it 1DROISpectrum.vi}

Take the 1-D power spectrum of the input signal. Data is passed
through a Hamming window and then zero padded before taking the
FFT. The output can be converted to dB or left as ``raw'' data.
Phase information is available if ever needed.

\item {\it 2DROISpectrum.vi}

Take the 2-D FFT of a ROI. A CAL file can be subtracted from the
end result.

\item {\it AOICoords.vi}

Determine the indices for the AOI selected with {\it
AOIParams.vi}. The program returns the row numbers, the start
column, and number of points to take for the AOI.


\item {\it AttenCorr.vi}

Small program to find attenuation correction as a function of
range. Default $\alpha$=0.5 db/cm~MHz. Can be used for both
parameter images or RF data. Output is an array to correct
midband, $2 \alpha x f_{c}$, and one to correct slope, $2\alpha
f_{c}$.

\item{{\it ColorEncode.vi}}

Overlay color for likelihood of cancer on parameter images. Window
size can be chosen.

\item {\it ExtractName.vi}

Extract the name and case number from the input file name. Also
check to see how many digits are used to specify the case number.

\item {\it FindLine.vi}

Find the indices of a line in sector scan ref frame and convert to
indices in the raw data ref frame. Can also use three input points
to convert a box in the sector format to its equivalent in the raw
data reference frame.

\item {\it FindSectorBox.vi}

Convert a box in sector format to one in the raw data system.
Essentially converting $(x,y)$ to $(r,\theta)$. Where $(x,y)$ is a
box in sector format and $(r,\theta)$ represents the raw data
where $r$ is RF point (range) and $\theta$ is scan line (angle).

\item {\it FitSpectrum.vi}

Fit the selected frequency band using linear regression and return
slope, intercept, mid-band fit, and error values.


\item {\it MakeEllipse.vi}

Find the coordinates of an ellipse based on the input arguments
taken from three cursors in a graph window. One cursor defines the
center of the ellipse and the other two the major and minor axis.

\item {\it MovableROIFit.vi}

Moves a ROI through a whole image and returns the fit parameters.
The fit parameters for a CAL file are also returned if needed. The
program can take both a single ROI or scan through a whole image.
User selects ROI size and number of boxes in both the $X$ and $Y$
directions or the ROI size and start coordinates.

\item {\it PIByteFiles.vi}

Create byte stream files from parameter image arrays. Geared
towards B\&K data right now. Byte files are scaled to the input
ranges for mid-band, intercept, and slope.


\item {\it ProstSpecFormat.vi}

Take data in {\it SpecCurveData.vi} and convert it to a text file
where each variable is a fixed character width. Needed to prepare
data for prostate data base.


\item {\it ROIAverage.vi}

Find average attenuation values for a region of interest in {\it
SectorBiospy.vi}. For Hitachi data, also find average cross
calibration for ROI.

\item {\it ScaleParamData.vi}

Scale data to a range of zero to some upper bound. Need to enter
max and min range values for the scaling.

\item{{\it ScatTheory.vi}}

Algorithm to compute $CQ^{2}$ or spectral parameters from
$CQ^{2}$.

\item {\it SpecCurveData.vi}

Assemble ROI and fit parameters into a string of characters which
can be written to {\it Spectra.txt}

\item {\it SubSpecOneCase.vi}

If looking at a single case, subtract the CAL parameters from the
raw fit region of the ROI spectrum. This VI is needed because the
output of the FFT and fit comes in a format to handle a sliding
window FFT. Need to extract the data for the single ROI case.

\item {\it UnscaleParamData.vi}

Scale 8 bit parameter images back to their true dB spectral
parameter values.

\item {\it ZeroPad.vi}

Pad zeros onto the data before taking an FFT. User selects a power
of 2 or the nearest power of 2.

\end{enumerate}

\section{ProstateDataAcq}
Many of the programs described here are similar to what is
discussed in Sec.~\ref{scan sec}. The main differences are that
the VIs in this section do not use any motion control hardware and
data acquisition is done with a GaGe board rather than with an
Acqiris board.

\subsection{Main programs}
\subsubsection{{\it CopyToZip.vi}}
A simple program that copies files from the Current Scans folder
to Previous Scans folder and also copies the files to a zip disk.
The program can also send an email message back to RRI if the
computer used to take the data is networked.

\subsubsection{{\it ProstateGetData.vi}}

Similar to {\it AcqControl.vi} in Sec.~\ref{acqcon}. This is the
master control for prostate data acquisition. Parameters are
entered that get saved in the EYE file. Data is acquired and saved
in RAM until a save to disk option is selected. Thumbnail images
of all the scans can be viewed as data acquisition is in progress.
In a research version of the software, more detailed options are
available that permit RTV calibration and data acquisition from a
single scan line across multiple frames.


\subsection{Subprograms}
\subsubsection{{\it Gage.llb}}
This library contains the VIs used to control the GaGe board. VIs
are also used from the library of programs installed with GaGe
software.

\begin{enumerate}

\item {\it GageSampInterval.vi}

Convert sampling rate to GaGe input parameter.

\item {\it GageTrigLevel.vi}

Convert trig level to GaGe input parameter.

\item {\it GageVolScale.vi}

Convert voltage scale to GaGe input parameter.

\item {\it SetGagePrams.vi}

Set up all GaGe input parameters.

\item {\it SetUpGage.vi}

Oscilloscope like program that acquires data from the GaGe board
and displays it. This VI is used in the research version of the
software to find the scan line with the maximum voltage. {\it
ProstateGetData.vi} does not use this VI in the clinical version
of the software.

\end{enumerate}

\subsubsection{ProstateScanSystem.llb}
\begin{enumerate}

\item {\it CountConfig.vi}

Set up counter on the NI 6601 counter/timer card to take a frame
of data or to acquire calibration data. Works with both a Hitachi
system or a B\&K system (3535 or Hawk).

\item {\it DigGageData.vi}

This is where the GaGe data is acquired and transferred to memory.
Parameters for each scan are also recorded along with the data.
TGC data can be saved if using a B\&K.


\item {\it FrameTrigger6601.vi}

For frame acquisition, the counter is set up to generate an output
pulse on the falling edge of a vector. The counter is armed by the
frame signal. For the Hitachi it comes from the machine; for the
B\&K it comes from another counter. The source input of the
counter is actually routed internally by the 6601 card to the gate
input. Hardware arming of the trigger occurs on PFI line 30. Note
that hardware arming is available on the dedicated counter/timer
cards from NI, but not on multi-purpose A/D cards.

\item {\it FTPData.vi}

This program allows data to be FTPed back to RRI. Also sends an
email message that data is being sent. Not very elaborate error
checking and has not been tested remotely yet.

\item {\it NumberMultFile.vi}

Add a scan number to root file name.

\item {\it ProstateParams.vi}

Set up a cluster of parameters for each scan. These are used to
generate the header.

\item {\it ProstateThumbnails.vi}

Create small thumbnails of each scan for quick viewing.

\item {\it QuitMode.vi}

Option window to verify user really wants to exit software. If FTP
is available, it also gives an FTP option.

\item {\it RemNoiseRec.vi}

Used for B\&K Hawk only. Subtracts out ``noise'' from the scan
lines. First two lines of each from are coherent noise. First line
for odd lines, second line for even lines.

\item {\it SaveGageData.vi}

Save all acquired data to eye files.

\item {\it ScanParams.vi}

All machine specific parameters are selected and kept track of in
this vi.

\item {\it VectorTrigger6601.vi}

The vector trigger is more straightforward than frame acquisition.
Vectors go to the source of the counter, the frame signal to the
gate. An output is generated by counting the number of vectors
after the falling edge of each frame sync. This allows an output
trigger to occur at the same vector sync for every frame. This
program will work with normal NI timer also, although it may need
to be modified slightly.

\end{enumerate}




\section{Scanning Control}
\label{scan sec}
\subsection{Main programs}

\subsubsection{{\it mastercontrol.vi}}

This VI acts as the control panel for the different functions of
data acquisition. Several options are presented: the motors can be
manually controlled, the motors can be moved to an ($X,Y,Z$)
position, the motors can be sent back to the ($0,0,0$) position,
or a scan mode can be selected. The options for motor control open
another control window to perform the desired function. To perform
a scan, the scan mode is chosen, and then the button marked {\bf
GO} needs to be pushed.


\subsection{Subprograms}

Some of the subprograms can nearly act as stand alone programs.
However, because some global arguments are used and parameters get
passed between subprograms, they will not normally function on
their own.


\subsubsection{{\it DP110.llb}}

This contains the controls for the Acqiris DP 110. The low level
programs are in the LabVIEW directory in the {\it Instr.lib}
folder.

\begin{enumerate}

\item {\it conMem.vi}

Set up segments and points per segment of digitizer.

\item {\it conSamp.vi}

Set up sampling rate and delay of digitizer.

\item {\it conTrig.vi}

Set up trigger of digitizer.

\item {\it conVert.vi}

Set up coupling and voltage scale of digitizer.

\item {\it RFWindow.vi}

Determine where to place plot cursor to indicate what data will be
captured.

\item {\it SampInterval.vi}

Convert numerical sampling rate control into an actual value which
can be input into the DP100.

\item {\it SetUpDigi.vi}

Set up the Acqiris digitizer  to take data. Used like an
oscilloscope. Can select voltage scale, delay, and time base. If
using a therapy burst, can also select those parameters. A red
line on the digitizer traced shows what data will be captured
during the scan.

\item {\it SingleSeq.vi}

Take a single sequence of data and return as a true voltage.

\end{enumerate}

\subsubsection{Panametrics}

The control of the 5900PR is largely taken from VIs provided by
Panametrics.


\begin{enumerate}

\item {\it 5800PRSet.llb}

Can act as stand alone program. Checks status of 5900PR and then
changes settings if needed.

\item {\it 5800PR.llb}

{\it 5800PRStatus.llb} check current settings, {\it 5900RW.llb}
can read or write to 5900, and {\it GPIB Error Report.llb} reports
an error if one came up.

\end{enumerate}


\subsubsection{{\it MM200.llb}}

This library contains programs to perform basic movement of the
motion system via the MM2000 motion control card. There is no need
to discuss them because they are named to reflect their function.


\subsubsection{{\it Ctrlpanl.llb}}

This library also contains programs to perform basic movement of
the motion system, and many of the features are duplicates of the
{\it MM200.llb} library. However, the subVI named {\it
MM2KCTRL.VI} is worth noting. It allows manual movement or
incremental movement of the motors. This VI forms the core of the
motion control accessed through {\it mastercontrol.vi}.

\subsubsection{{\it ScanSystem.llb}}
\label{acqcon}

This library contains the meat and bones of the scan system. As
mentioned earlier, the programs are able to operate as stand alone
units to only some extent because global variables are used.

\begin{enumerate}

\item {\it 33120PRF.vi}

Set the frequency on a HP33120A function generator. Used to set up
the prf signal on the SYNC output

\item {\it 33123oSyncStat.vi}

Allows the SYNC output of a HP33120A function generator to be
turned off and on.

\item {\it AcqControl.vi}

This vi provides the options for performing a scan. The look of
the front panel changes depending on if it is a 1-D or 2-D scan.
The appropriate parameters are available for the user to input.
When the parameters are ready for use, a scan is commenced by
pushing {\bf Start Scan}. Pushing {\bf Exit} ends the execution of
the program.

Global variables are extensively used. This helps reduce the
clutter of wiring and passing arguments between subprograms. The
program starts by setting up the PRF source which is an external
function generator. The user can select an option to set up the
Panametrics P/R ({\bf Set Panametrics}). The digitizer ({\bf Set
Digitizer}) needs to be set up before taking data. More detailed
instructions on the use of the software available.

\item {\it CaseParams.vi}

Assembles a cluster of arguments for each scan performed. Used so
to keep track of parameters when saving data.

\item {\it ContMotion.vi}

This vi assembles a program to perform a continuous motion scan
with triggers at specified intervals. Used by {\it
SendProgram.vi}.

\item {\it DigData.vi}

This VI controls the actual scan and data acquisition. First, the
counters are stopped if using an internal PRF or if the therapy
transducer is being used. Then the Acqiris is set up to take a
batch of sequence data. Counters are then started or the motors
are moved to provide the PRF triggers for the Acqiris. Motion can
either be a continuous sweep or in incremental steps, depending on
the type of scan. After the data is acquired, the counters are
again stopped, the therapy voltage, if used, is set to zero and
the motors are returned to their start position.  The data is then
assembled into a 2-D matrix with each column a single scan line.


\item {\it EXTPRFCount.vi}

This VI sets up the counters to divide down the PRF and to send
out a delayed pulse to the therapy transducer. Counter 0 provides
the divide down and counter 1 provides the delay.

\item {\it IncSteps.vi}

This vi assembles a program to perform an incremental motion
triggers at a specified increment. Used by {\it SendProgram.vi}.

\item {\it InputHeader.vi}

Panel that opens prior to saving EYE files. Contains the fields in
the header and permits changes to be made to some of them.

\item {\it IntPRFCount.vi}

Sets up counters on the A/D board to generate a PRF. Counter 0 is
the PRF and counter 1 divides down the signal. The divided signal
can trigger the therapy. This VI is no longer used. Instead {\it
EXTPRFCount.vi} is used because the PRF is always from an external
source. If setting up a system with no motion control, this VI
could be useful.


\item {\it MM2KManCTRL.VI.vi}

Allows manual control of the motors. By depressing the controls,
motor can be moved in any direction. The position is updated as
the motor moves.

\item {\it MotionGlobals.vi}

This VI contains all the global variables. It is used by many of
the programs in this section.

\item {\it MoveToXYZ.vi}

Moves motors to the specified ($X,Y,Z$) position.

\item {\it NumberFileName.vi}

Used for labeling 2-D scans. Allows a number to be added to the
case name.


\item {\it SaveData.vi}

Save the data from a scan in eye files. Uses arguments assembled
in {\it CaseParams.vi} to set up values in the header. User can
alter any header parameter and then save the data. If a 2-D scan
was used, data files are automatically numbered.

\item {\it SendProgram.vi}

Sends a motion program to the MM2000. The program is assembled
based on input parameters sent to the motion board and is then
compiled by the MM2000 card. Two types of programs can be sent,
one for a continuous motion with trigger pulses at specified
intervals, or one for incremental motion with a trigger pulse at
each incremental move.

\item {\it SetPid\&FindPos.vi}

Set the parameters for the motors such as velocity. Also find the
current position and write it to the global variables.

\item {\it ShowThumbs.vi}

Displays a 1-D array of Picts as a 2-D array. User specifies
number of columns.

\item {\it Therapy\&Scan.vi}

Allows therapy to be fired for longer durations than just 2000
cycles of HP burst mode. Calls {\it Therapy.vi} between taking
A-Scans. User specifies A-scans before and after therapy.

\item {\it Therapy.vi}

Allows therapy transducer to be fired for some specified number of
seconds. Makes a call to {\it TherapyBurst.vi} to set up the
function generator in a gated mode. Counters 0 and 1 are used to
gate the therapy function generator for the desired length of
time.


\item {\it TherapyBurst.vi}
Sends GPIB commands to adjust the settings of the function
generator used to control the therapy transducer.

\item {\it Thunmbnails.vi}

Make thumbnail images of all the scans. The thumbnails can be
viewed as an array by selecting the appropriate option where
available.

\end{enumerate}


\section{TransducerCal}


\subsection{Main programs}

\subsubsection{{\it BeamCal.vi}}

Takes EYE files from pulse measurements with a hydrophone and
computes beamwidth and other parameters. Some of the parameters
need to be manually entered.

\subsubsection{{\it FindBandwidth.vi}}

Open a cal fine (two column format) and determine bandwidth and
center frequency.

\subsubsection{{\it PulseCal.vi}}

Allows calibration of pulses from a transducer using the FDA
definitions of various parameters. Need to enter a few parameters
from when the data was taken. Then can take an EYE file and select
a scan line that contains the pulse information. By selecting the
region of the line that has the pulse, all the relevant parameters
are generated.

\subsubsection{{\it TestTrans.vi}}

Will perform a 1-D or 2-D scan and allow for characterizing the
beam pattern of a transducer. Uses many of the same VIs that were
found in Sec.~\ref{scan sec}. Data will be presented in both a 3-D
form and also 2-D. Still need to work out details.

\subsection{Subprograms}

\subsubsection{{\it CalTransducer.llb}}


\begin{enumerate}
\item {\it ChoosePram.vi}

Finds selected parameter for scan data. Can find pos, neg,
peak-peak, and rms voltages or single pulse parameters.

\item {\it Derate\&MI.vi}

Find derated parameter values for ultrasound pulse. Takes inputs
from {\it IntParams.vi}.

\item {\it FindHydSens.vi}

Find a sensitivity of a hydrophone for a specified frequency. Can
choose a calibration from one of our hydrophones.


\item {\it IntParams.vi}

Finds the intensity parameters such as $I_{ta}$ and $I_{pa}$ for
an ultrasound pulse.

\item {\it MoveandWait.vi}

Move motor and wait for specified time duration.

\item {\it MultiPlane.vi}

Used when taking 3-D data of a transducer. Allows for multiple
scan plane data sets by incrementing up (vertical) in the
z-direction after each scan and saving the data to a spreadsheet
file.

\item {\it PulsePrams.vi}

Finds pulse from an RF signal and computes the pulse parameters
using {\it FindHydSens.vi}, {\it IntParams.vi}, and {\it
Derate\&MI.vi}.

\item {\it RapidScan.vi}

Takes data in a continuous sweep rather then in incremental steps.
Speeds up the data collection.

\item  {\it ScanTrans.vi}

Performs scans for {\it TestTrans.vi}. Moves motors along desired
axis in an incremental fashion, stopping briefly at each point to
take data. Idea is to use a hydrophone to sample a CW wave from a
source and then to extract parameters from the wave to
characterize the source. May want to allow a burst from source.
Can use a pre-trigger or delay on Acqiris to take the data
properly. Will need ability to display trace from Acqiris.

\end{enumerate}

\section{3DdataPrep}
\subsection{Main programs}
\subsubsection{{\it 3DVisPrep.vi}}

This program allows data stored in scan planes (either spreadsheet
format or RGB BMP) to be converted to several types of output
files. The output options include conversion to 8 bit byte files
for each plane or a single byte file with all the data. The
spreadsheet data can also be converted to 8 bit BMP files.

\section{Annular Array}
\subsection{Main programs}

An assortment of programs to compute the SIR from an annular
array.

\end{document}
