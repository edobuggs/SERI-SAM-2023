\documentclass[10pt]{article}
\usepackage{srcltx,graphicx}

\textwidth 6 in \textheight 8.75 in

\voffset-.75in \oddsidemargin.25in \evensidemargin.25in

\begin{document}

\begin{center}
 {\huge \bf Riverside Research Institute}
\end{center}

{\bf To:} Biomed group

{\bf From:} Jeff Ketterling

{\bf Date:} \today

{\bf Subject:} Condensed summary of scan system use.
\\


\begin{enumerate}
\item (This only applies to the old MM2000 system at RRI).
Login to account ``biomed" with password 1234.

\item Turn on all equipment before running the software. This is
particularly important for the amplifier used to drive the motion
stages.

\item Check for air bubble on the transducer face.

\item Set up digitizer before acquiring any data. Remember that the
distance seen when setting up the digitizer is the round trip
distance.

\item Run the software, move the stages, take scans, etc.

\item Be careful not to perform a motion scan that will cause a motor to
hit one of its limit switches. If this happens, the stage will
have to moved to a position with enough clearance before taking a
scan.

\item (This only applies to the old MM2000 system at RRI). If a
repetitive beeping sound can be heard coming from the computer,
stop the execution of the software and push the reset button on
the motion control card in the computer. (A small black button on
the lowest most card in the computer.) Then run the software
again. If this does not eliminate the beeping, then reboot the
computer.

\item (This only applies to the old MM2000 system at RRI). If a scan
does not finish properly. Close the software, push the reset
button on the motion control card, and try the software again.

\end{enumerate}

\end{document}
