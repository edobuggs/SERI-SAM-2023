\documentclass[10pt]{article}
\usepackage{srcltx,graphicx}

\textwidth 6 in \textheight 8.75 in

\voffset-.75in \oddsidemargin.25in \evensidemargin.25in

\begin{document}

\begin{center}
 {\huge \bf Riverside Research Institute}
\end{center}

{\bf To:} Biomed group

{\bf From:} Jeff Ketterling

{\bf Date:} \today

{\bf Subject:} How to use new scanning system.
\\

This memo discusses the operation of the new automated scanning
system. The focus is on how to operate the software, rather than
technical aspects of the how the overall system works. For a
description of how the software works, examine the block diagrams
of the LabVIEW VIs. Most have notes indicating what is being done.
The computer used to control the apparatus requires the
account``biomed'' with a password 1234.
%\\ \hline

\section{Wiring Connections}

The BNC cable wiring schematic for the scanning system at RRI is
shown in Fig.~\ref{fig:BNCWires}(a). All of the equipment is
either mounted on a relay rack or in the controlling computer. The
switch box permits software selection of a PRF from either a motor
or a function generator. Cables connect cards in the computer to
the appropriate components in the relay rack. These include
connections between a Motion Master 2000 or National Instruments
motion control card and a break out box, a GPIB card and several
pieces of equipment (Fig.~\ref{fig:BNCWires}(a)), and a
multipurpose data acquisition card and a break out box. A trigger
line and RF line also go to an Acqiris DP110 A/D card. Additional
cables link the three DC motors to a driver box on the relay rack.

A reduced version system without a therapy component is shown in
Fig.~\ref{fig:BNCWires}(b). This system is currently set up in
Hawaii. Both systems use the same operating software except that
the Hawaii system has fewer available options. The national
instruments software will work with both stepper and DC motor
systems.


\begin{figure}[htb]
\begin{center}
\includegraphics[width=3.5 in]{BNCWiring.eps}
 \caption{Wiring diagram for equipment.}
 \label{fig:BNCWires}
\end{center}
\end{figure}

The switch box provides signal connections between the 6024E and
Motion Master 2000 cards (Fig~\ref{fig:BNC2090}) and is only used
at RRI. The box three inputs (the function generator PRF) and two
outputs (a digitizer/panametrics trigger and a therapy
gate/trigger). The connections shown in Fig~\ref{fig:BNC2090} that
refer to a pin number and wire color are for the auxiliary
connector of the Motion Master 2000 interface box, the numbers
toward the bottom of the figure refer to pins on the 6024E. The
connection is made via a DIN cable that hangs out of the box. The
motor trigger BNC is used only for NI based motion control and the
DIN cable should be disconnected in this case.



\begin{figure}[htb]
\begin{center}
\includegraphics[width=2.5 in]{BNC2090.eps}
 \caption{Wiring diagram for PRF source switching box. A pin
 number plus wire color refers to the Motion Master 2000 connections
 card. The connections are made via a 15 pin DIN cable than connects to
 AUX connector of the MM2000 axis break out box.The numbers at the bottom
 of the figure refer to pin numbers of the National Instruments 6024E card.
 A BNC input for the motion trigger is used for NI systems}
 \label{fig:BNC2090}
\end{center}
\end{figure}

Inside the switch box, the PRF signals from the function generator
and motor first pass through TTL circuits. The circuits are a
SN7405 hex inverter for the motor PRF pulse, and a SN54122
monostable multivibrator for the function generator PRF source.
The SN7405 acts as a protection buffer for the Motion Master 2000
I/O line. The SN54122 reduces the PRF duty cycle from the HP33130A
to avoid trigger problems with the Panametrics 5900PR (refer to
5900PR manual for more info). A two line multiplexer (SN54157) is
then used to select either the function generator or the motor
pulse as the PRF source used to trigger the A/D card. The
selection is done via software on line 2 of the 6024E card.

\section{Using the Scan System}

The operation of the scan system is controlled through the vi
named {\it mastercontrol.vi} (Fig~\ref{fig:mastercon}). {\it
Mastercontrol.vi} is a portal to all components of data
acquisition and motion control. {\bf Before running {\it
mastercontrol.vi}, turn on all electrical components of the scan
system (HP function generators, Panametrics, and motor power
supply), except for the ENI amplifier.} The type of motion system
in use should be selected and if it is not an available option, it
has been hardwired into the program. The system is now ready to
use and the vi is executed by pushing
\includegraphics{runvi.eps} in the upper left corner of the vi.

\begin{figure}[htb]
\begin{center}
\includegraphics[width=2.5 in]{mastercon.eps}
 \caption{Image of the {\it mastercontrol.vi} control panel.}
 \label{fig:mastercon}
\end{center}
\end{figure}

The options in the green box labeled Motion Control allow for
movement of the translation stages. The pull down menu above the
{\it Go} key selects what data acquisition mode is used. After
choosing a mode, pushing {\it Go} activates the selected mode.
Pushing {\it Exit} terminates execution of the program.

Occasionally, the Motion Master 2000 Board will not respond upon
initialization, causing the software to hang up. When this occurs,
the vis won't respond when options are selected on the front
panel. If this should happen, force termination of the program by
pressing the stop button (\includegraphics{stop.eps}). Close all
open vis except for {\it mastercontrol.vi} and then execute {\it
mastercontrol.vi} again.

The operation of {\it mastercontrol.vi} will be discussed in terms
of its subsystems. The subsystems are usually separate vis but
they are not meant to be used as stand alone programs. As the
overall program is used, various other vi windows will open as
options are selected. {\bf It is important to remember that once
done with these new windows, they should be closed by pushing the
red {\it Done} button. Do not force the vi window closed or simply
return to the original vi with a mouseclick.}

The remaining subsections discus the operation of the {\it
mastercontrol.vi} and its subsystems roughly in the order that the
options will be encountered.

\subsection{Positioning the Transducer}

The first options after executing {\it mastercontrol.vi} are for
motion control. They are in the green box labeled Motion Control.
The first two options ({\it manual motion control} and {\it move
to X-Y-Z}) open new vis. The third option, {\it Return to Home},
returns the translation stages to the $(X,Y,Z)$=(0,0,0) position.

Pushing {\it manual motion control} opens the vi named {\it
MM2KManCTRL.VI} (Fig.~\ref{fig:manCTRL}(a)). This vi allows for
manual movement of the translation stages. Each axis has a set of
controls allowing continuous or incremental movement. The current
position is actively updated as motion takes place.

\begin{figure}[htb]
\begin{center}
\includegraphics[width=4.5 in]{manCTRL.eps}
 \caption{(a) Control panel for {\it MM2KManCTRL.VI}. Depressing
 {\it minus} or {\it plus} moves the translation stages until the
 button is released. The circular buttons move the stages in 1 cm
 increments. {\it RST} makes the current position the zero position.
 (b) Control panel for {\it MoveToXYZ.vi}. Entering an absolute
 position into the position indicator moves the motor to that position.
 Pushing {\it Rst} resets the position indicator to zero. (c) New software
 that has both functions combined.}
 \label{fig:manCTRL}
\end{center}
\end{figure}

Continuous motion is accomplished by depressing the buttons marked
{\it minus} or {\it plus}. The translation stage for that axis
will move while the button is pressed. Once released, the stage
will stop. If the blue circular buttons are pushed, the stage will
move in 1 cm steps. To reset to the zero position, simply push the
black button {\it Rst}. When through positioning the stages,
select {\it Done}. {\it MM2KManCTRL.VI} will then close and you
will return to {\it mastercontrol.vi}.

Pushing {\it move to X-Y-Z} opens another vi named  {\it
MoveToXYZ.vi} (Fig.~\ref{fig:manCTRL}(b)). This vi allows movement
based on the input co-ordinates. Entering new co-ordinates moves
the motor to the new position. Like before, the position can be
reset to zero at any time and when through with the program,
pressing {\it Done} returns to {\it mastercontrol.vi}.

Note that the maximum translation for the $X$ and $Y$ stages is 10
cm and for the $Z$ stage 5 cm. If the motor moves too far, it will
activate a stop switch and the motor will wait until it is moved
back to a valid position. {\bf Therefore, it is important to make
sure enough space is available before taking any B scans.}

\subsection{Scan Modes}

Once the translation stages are positioned, a scan mode can be
entered. The scan modes are entered by selecting an option from
the pull down menu above the {\it Go} button
(Fig.~\ref{fig:scanmenu}). Except for the first scan mode option,
all the modes take scan line data and use the same vi. Each mode
will be discussed separately since the options change slightly for
each mode.


\subsubsection{Therapy With No Scan}
\label{sec:thernoscan}

This mode allows the therapy transducer to be fired for a selected
number of seconds. No scan lines are taken in this mode. When the
mode is selected, a new vi opens up {\it Therapy.vi}
(Fig.~\ref{fig:therapyvi}). This vi permits a selection of the
therapy frequency, amplitude, and how long the therapy fires. Once
these values are chosen, selecting {\it Start Therapy} activates
the therapy transducer. {\bf To stop the therapy burst in the
middle of operation, you can either press {\it Emergency Stop} or
the Esc key on the keyboard.} When finished with this mode,
pushing {\it Done} returns to {\it mastercontrol.vi}.

\begin{figure}
\begin{minipage}[t]{0.5\linewidth}
\centering
\includegraphics[width=2in]{scanmenu.eps}
 \caption{Scan menu in {\it mastercontrol.vi}. After selecting a scan mode,
 pressing {\it Go} activates that modes control vi.}
\label{fig:scanmenu}
\end{minipage}%
\begin{minipage}[t]{0.5\linewidth}
\centering
\includegraphics[width=2in]{therapyvi.eps}
\caption{Control panel for {\it therapy.vi}. After selecting
parameters in the green box, pushing {\it Start Therapy} fires the
therapy transducer for the selected amount of time. Pushing {\it
Emergency Stop} or Esc will stop the therapy signal.}
  \label{fig:therapyvi}
\end{minipage}
\end{figure}

\subsubsection{Therapy With Scan}
\label{sec:therscan}

The Therapy With Scan mode allows for a pre and post look M-Scan
when the therapy transducer is fired. When this mode is selected,
{\it AcqCont.vi} opens with options for this particular mode
(Fig.~\ref{fig:AcqTher}). {\it AcqCont.vi} will be used by all the
remaining modes with slightly different input parameters for each
mode. Therefore, the discussion will be a bit more detailed in
this section to introduce the common features used in all the
modes. The general layout is for the the control features to be on
the left and the display features on the right. The discussion
will follow a counter clockwise route for how to operate the vi.

\begin{figure}[!htb]
\begin{center}
\includegraphics[width=5.8 in]{AcqTher.eps}
 \caption{Control panel for {\it AcqCont.vi}. The controls on the left side
 relate to input parameters and the controls on the right side to viewing
 and saving acquired data. The left side controls will vary depending on what scan
 mode is selected.}
 \label{fig:AcqTher}
\end{center}
\end{figure}
\begin{figure}[!htb]
\begin{center}
\includegraphics[width=3.8 in]{5900.eps}
 \caption{Control panel for {\it 5900PRSet.vi}. This vi accesses the same
 option that are available on the front panel of the 5900PR.}
 \label{fig:5900}
\end{center}
\end{figure}
\begin{figure}[!htb]
\begin{center}
\includegraphics[width=3.5 in]{header.eps}
 \caption{Control panel for {\it SaveData.vi}. This vi allows the
 user to select parameters in the EYE file header before EYE files
 are saved.}
 \label{fig:header}
\end{center}
\end{figure}

The {\it prf (kHz)} selects the prf from the HP33120A function
generator (GPIB 11). Next to it, the equivalent time for the prf
in ms is given. Below this control is a green box which controls
the Therapy Parameters. These parameters are the same as in
Sec.~\ref{sec:thernoscan} with two additional options: {\it Lines
Before} and {\it Lines After}. These options represent how many A
scans to take before and after the therapy burst.

Below the Therapy Parameters box is a button to {\it Set
Panametrics}. When this is chosen the vi {\it 5900PRSet.vi} opens
(Fig.~\ref{fig:5900}). The options in this vi are the same as
those available on the 5900PR. Changes can be made directly on the
instrument without effecting any of the vi discussed here. {\it
5900PRSet.vi} is provided because it is somewhat more convenient
to use than the actual front panel of the instrument. Once the
settings of the 5900PR are selected, push {\it Done} to return to
{\it AcqCont.vi}.

The next control, {\it Set Digitizer}, is discussed in more detail
in Sec.~\ref{sec:digi}. Before taking data for the first time,
this option needs to be selected. A new vi will open, allowing a
rf sampling window to be selected. Once the digitizer is set up,
data can be acquired by pushing {\it Start Scan}.

Once {\it start scan} is selected, data is acquired and displayed
in the figure on the right side of {\it AcqCont.vi}. Under the
plot are two controls. One allows the data to be represented as a
linear compression or a log compression. The other control allows
the axis to be scaled to a distance, rather than a count.

The controls above the plot select the naming convention of the
files and allow data to be saved. {\it Case name} selects the root
name for the files. It can be changed at anytime. As data is
taken, a scan number is appended to the case name, always starting
with 1 for a new case name. Previous scans can be viewed by
selecting the appropriate case name in the pull down menu below
the {\it Save All Data} button. They can also be viewed as a
collection of thumbnail images by selecting {\it View Thumbnails}.
When this option is selected, another vi opens up in the
background ({\it Thumbnails.vi}). {\it Thumbnails.vi} will display
thumbnail scan images with an overlaid file name as data is taken.

When the data is ready to be saved, pressing {\it Save All Data}
initiates the process. If you do not want to save all of the
cases, press the red {\it keep?} button next to the appropriate
case so that it is not lit. {\bf You must choose which cases not
to save before pressing {\it Save All Data}}. The default setting
for each new file is to save it. A new vi opens up
(Fig.~\ref{fig:header}) showing all the header information that
will be saved with the EYE file. Most of the fields can be changed
at this point except for items that were fixed upon scanning
(Sample Rate, \# rf points per line, RF file name, Patient ID,
etc). These items are grayed out to indicate they cannot be
changed. When ready to save the files press the button labeled
{\it Push here when finished with header info} at the bottom of
the vi. All the data is saved to the D:$\backslash$Temp directory.

To delete all of the acquired data, press {\it Erase All Data}. To
enter a different scan mode or to exit the program, press {\it
Done} and return to {\it mastercontrol.vi}.

\subsubsection{M Scan}

The M scan mode differs from the above section only in the choice
of input parameters. Figure \ref{fig:mscan} shows the upper left
corner input parameters available in {\it AcqCont.vi}. The choices
now are just for {\it prf} and {\it lines per scan}. Digitizer
settings may also need to be changed, but that is discussed in a
later section.

\begin{figure}[htb]
\begin{center}
\includegraphics[width=2 in]{mscan.eps}
 \caption{Input parameters for {\it AcqCont.vi} when in M Scan mode.}
 \label{fig:mscan}
\end{center}
\end{figure}

\subsubsection{M Scan With Therapy}

The options in {\it AcqCont.vi} look the same as in
Fig~\ref{fig:mscan}. The therapy parameters are chosen in the vi
activated by selecting {\it Set Digitizer}. A green light will
indicate that the total number of data points is valid. If the
indicator turns red, the number of RF points of number of scan
lines needs to be reduced until the display turns green again. The
therapy mode for this and in the next sections now differs
slightly from the ones in Secs.~\ref{sec:thernoscan} and
\ref{sec:therscan}. The therapy is now slaved to a divide down
circuit linked to the prf. A divide down factor ($n$) is chosen
for the prf, and then the therapy is fired every $n$ triggers.
Unlike before, the therapy is limited to a maximum of 50000
cycles.

\subsubsection{1-D B Scan}

As in the previous mode, the input parameters change slightly when
1-D B Scan is selected (Fig.~\ref{fig:1dBscan}). Now you choose
the spacing between scan lines instead of the prf. The total
distance of the scan is displayed based on the {\it lines per
scan} and {\it Distance between scan lines ($\mu$m)}. The prf
source will now be derived from the motors. All previous modes
used the function generator as the prf source.


\begin{figure}[htb]
\begin{center}
\includegraphics[width=2 in]{1dBscan.eps}
 \caption{Input parameters for {\it AcqCont.vi} when in 1-D B Scan mode.}
 \label{fig:1dBscan}
\end{center}
\end{figure}


\subsubsection{1-D B Scan With Therapy}

This mode looks just like 1-D B Scan except now a therapy burst
can be fired with a divide down setting.

\subsubsection{2-D B Scan}

In the 2-D B Scan mode, a few more parameters than in the 1-D B
Scan mode are available. Now the user must select {\it number of
scans} and the {\it Distance between scans ($\mu$m)}
(Fig.~\ref{fig:2dBscan}). These settings determine the number of
scan planes and the distance between the scan planes.

\begin{figure}[htb]
\begin{center}
\includegraphics[width=4.5 in]{2dBscan.eps}
 \caption{Input parameters for {\it AcqCont.vi} when in 2-D B Scan mode.
 A control is now available to select which scan plane to view.}
 \label{fig:2dBscan}
\end{center}
\end{figure}

This case differs slightly from the earlier modes because each
scan generates multiple data sets. The {\it case name} now
represents a 2-D scan where each case has multiple scan planes. A
number is still appended to the Case Name for each new scan, but
now a letter {\bf a} is added to the end of it. (Case1a is the
first scan, Case 2a is the second, etc.) To view the individual
scan planes for each case, the {\it Scan to View} control is
incremented to the appropriate scan plane number. Note that the
number starts from 0 not 1. To view a previous case, the menu
control below {\it Save All Data} is still used. Now when the data
is saved, the data from each case retains its case name (Case1a)
and the scan plane number is appended to the end of this name. For
example, Case2a003 would be the third scan plane of Case2a.


\subsubsection{2-D B Scan With Therapy}

This mode looks just like 2-D B Scan except now a therapy burst
can be fired with a divide down setting.

\subsubsection{Rotational Scan}

This mode uses a fourth axis that rotates a specimen while
scanning. The probe remains fixed while the specimen rotates. This
mode also permits 2-D scanning in multiple z-axis planes. This
mode is only available with the NI based systems. The mode permits
the scan angle to be selected in 45 $^{\circ}$ increments and the
number of scan lines are selected with a pull down menu. A red
light will turn if an invalid combination is used or if the number
of data points exceeds the hardware capability.

\subsection{Setting up Digitizer}
\label{sec:digi}

When the {\it Set Up Digitizer} control is selected in {\it
AcqCont.vi}, the vi {\it SetUpDigi.vi} opens
(Fig.~\ref{fig:SetDigi}). {\it SetUpDigi.vi} allows the user to
select a RF window for digitization. The vi functions much like an
oscilloscope but with a few additional control parameters. The vi
is divided into three zones, Digitizer Settings, Therapy Burst Set
Up, and an active display of the RF signal.

\begin{figure}[!htb]
\begin{center}
\includegraphics[width=5.5 in]{SetDigi.eps}
 \caption{Control panel for {\it SaveUpDigi.vi}. This vi selects what
 region of the RF signal to capture. The data between the yellow and red
 cursors is the data that will be digitized. {\it Sampling Rate} and {\it rf
 points per line} select the ``length'' of data. {\it Ref Delay ($/mu$s)} and
 the position of the left cursor select the first data point. Therapy settings
 are only available if a Therapy scan mode was selected.}
 \label{fig:SetDigi}
\end{center}
\end{figure}

The Digitizer Settings are all enclosed within a green border
which contains several sub-units, also enclosed in green boxes.
The RF data Sampling parameters select what region of the active
display will be digitized in {\it AcqCont.vi}. {\it Sampling Rate}
and {\it rf points per line} select the data region to digitize
which is represented by the data between the Yellow and Red
cursors in the active display. {\it Ref Delay ($\mu$s)} selects a
delay relative to the PRF trigger and selects the first point in
the active display. The {\it Ref Delay ($\mu$s)} combined with the
location of the Yellow cursor result in a value for {\it
Equivalent delay distance (mm)} and {\it Total Delay ($\mu$s)}.

The remaining Digitizer Settings function much like the controls
on an oscilloscope. Scale Settings select the vertical and
horizontal parameters of the Acqiris DP110. {\it Voltage Scale}
represents the full digitization range of the DP110 with 5
V$_{pp}$ being the maximum allowable parameter and 50~mV$_{pp}$
the minimum. {\it Coupling} represent the channel coupling and
needs to be either DC 50 $\Omega$ or AC 50 $\Omega$. {\it Time
Scale per Div} represents time per division in the active display
with a total of ten divisions. Therefore the total duration of the
displayed RF signal is 10 times the value of {\it Time Scale per
Div}. The final Digitizer Settings are for the trigger source and
will normally not need to be adjusted. The default values are for
an external, 1 M$\Omega$ DC coupled source that triggers the DP110
on its rising edge with a 100 mV level.

The text box labeled {\it Error Message} will normally say
something like {\bf Triggered Successfully}. This indicates that
the DP110 is functioning properly. If some other message appears,
then a problem may be present in the system. If a message saying
{\bf Triggers Missing} appears, then the trigger source on the
switch box is likely set to Motor PRF or a BNC cable is not
connected properly.

The active display will show the RF signal and the data region
selected for digitization. The Red cursor is slaved to the Yellow
cursor based on the values of {\it Sampling Rate} and {\it rf
points per line}. The Yellow cursor can be moved by selecting it
with the mouse and dragging it to a new position. The scales for
the active display are in mV and units of the selected {\it Time
Scale per Div}. The spectrum of the full RF signal can be viewed
by selecting {\it View Freq Domain} rather than {\it View Time
Domain}. When a spectrum view is chosen, the axis need to be
re-scaled by pushing \includegraphics{autox.eps} and
\includegraphics{autoy.eps} in the control below the active
display.

The Therapy Bust Set Up controls are only displayed if a therapy
scan mode is being used. The therapy is limited here to a maximum
of 50000 cycles. The therapy is fired every $n$ counts of the PRF
as selected in the {\it Divide prf} control. The therapy delay
represents a delay in the therapy burst relative to the PRF
trigger. {\it Therapy frequency (MHz)} and {\it Therapy amplitude
(mV)} are just what they appear to be. The therapy bursts can be
momentarily stopped by pushing {\it Pause Therapy}. The control
will then read {\it Therapy Off} and when it is pushed again, the
therapy bursts will return.

Once a region of the RF signal has been selected for digitization,
pushing {\it Done} returns to {\it AcqCont.vi} and scan data can
be acquired.

\section{Transducer Characterization}

The vi {\it TestTrans.vi} (Fig.~\ref{fig:TestTrans}) operates in a
similar fashion to {\it manualcontrol.vi}, but is designed to
characterize a transducer beam. The main difference between the
two programs is that {\it TestTrans.vi} extracts a parameter from
each scan line (such as negative peak voltage) and then displays
it. The controls in Fig.~\ref{fig:TestTrans} will be discussed
from left to right, focusing mainly on features not yet discussed.
{\bf The connections change slightly because the DP110 Channel 1
input is now meant to be from a hydrophone.}

\begin{figure}[!htb]
\begin{center}
\includegraphics[width=5 in]{TestTrans.eps}
 \caption{Control panel for {\it TestTrans.vi}. This vi allows for the
 characterization of a transducer beam. Both 1-D and 2-D scans are possible.
 The vi is designed to work with a calibrated hydrophone, permitting the voltage
 data to be converted to pressure. The transducer can be operated in continuous
 wave, burst, or single pulse modes.}
 \label{fig:TestTrans}
\end{center}
\end{figure}
\begin{figure}[!htb]
\begin{center}
\includegraphics[width=3 in]{BurstWires.eps}
 \caption{Wiring diagram for equipment when using {\it TestTrans.vi}
 with a therapy transducer and hydrophone. The function generator is the one
 with GPIB address 11.}
 \label{fig:BurstWires}
\end{center}
\end{figure}

There are two options for the prf. The first mode is when the
5900PR is used to examine the pulse from diagnostic transducer.
This mode is selected when the control found under the {\it Exit}
button reads {\it Manual PRF}. In this mode the HP33120A with GPIB
address 11 is manually controlled to originate a PRF signal. The
system wiring is like in Fig.~\ref{fig:BNCWires} except the input
for Channel 1 on the DP110 is from the output of the hydrophone
pre-amplifier instead of from the RF output of the 5900PR.

The second mode of usage of {\it TestTrans.vi} is when a therapy
type transducer is being characterized. For this mode, push the
{\it Manual PRF}~\ button so that it reads {\it Burst Mode}. The
PRF still comes from the SYNC output of the HP with GPIB address
11 but now the output is a tone burst, generated by HP with GPIB
address 10, is used to drive a therapy type transducer. The
settings for the tone burst are selected via the vi rather than
directly on the function generator. The system BNC wiring changes
for this case are shown in Fig.~\ref{fig:BurstWires}.

Like before, the input to Channel 1 of the DP110 is connected to
the hydrophone output. The burst mode parameters are selected in
{\it Set Up Digitizer} (Fig.~\ref{fig:SetDigi}). Only the Therapy
Burst Set Up options for {\it Cycles, Therapy frequency,} and {\it
Therapy amplitude} will have any effect. Also make sure the
Trigger Settings has a Negative slope selected and that the
coupling is set for AC 50 $\Omega$. This will ensure that the
DP110 triggers at the start of a burst rather than the end of a
burst.\footnote{The SYNC HP33120A drops low when the burst takes
place, thus the need for a negative edge trigger.}

Like for {\it manualcontrol.vi}, {\it TestTrans.vi} is started by
pushing \includegraphics{runvi.eps}. After executing the vi, the
Motion Control options are available with one new addition: {\it
Go to Max}. {\it Go to Max} moves the transducer to the maximum
parameter values after taking a 1-D or 2-D scan. This provides a
means of moving the transducer to the position of maximum acoustic
pressure. {\bf Only select {\it Go to Max} once after each new
scan}.

Before taking data, a region of the RF signal needs to be selected
for digitization by using {\it Set up Digitizer}
(Sec.~\ref{sec:digi}). The parameters of the 5900PR can also be
adjusted with {\it Set Panametrics}. Now a transducer scan can be
taken. Select a {\it Scan Length (mm)} and {\it Number of pts per
scan} and the type of scan. The type of scan is chosen with a pull
down menu (Fig.~\ref{fig:scantype}). The start position of the
transducer is always the center point for any scan. Pushing {\it
Scan} will start the scan. The PRF source from the switch box can
either be from the function generator or from the motor PRF.

\begin{figure}
\begin{minipage}[t]{.45\textwidth}
\begin{center}
\includegraphics[width=1.5 in]{scantype.eps}
 \caption{Pull down menu for scan type in {\it TestTrans.vi}}
 \label{fig:scantype}
\end{center}
\end{minipage}
\hspace{1 cm}
\begin{minipage}[t]{.45\textwidth}
\begin{center}
\includegraphics[width=1.5 in]{pramsel.eps}
 \caption{Pull down menu for scan type in {\it TestTrans.vi}}
 \label{fig:pramsel}
\end{center}
\end{minipage}
\end{figure}

Once a scan is taken, the data is displayed in the plot of {\it
TestTrans.vi}. If the scan was 1-D, then a normal 2 axis plot is
displayed. If it was 2-D, then a 3-D plot is used. The axis of the
plots are in mm. The vertical (intensity) parameter is chosen with
the pull down menu labeled {\it Parameter Selection}
(Fig.~\ref{fig:pramsel}). The vertical parameter is always shown
as an absolute value and can be actively changed by changing {\it
Parameter Selection}.

The first four options of the parameter selection (Neg Peak, Pos
Peak, Peak to Peak, RMS) are exactly what their name implies. The
remaining parameters relate to definitions in the FDA 510(k)
guidelines and require inputs from the {\it Pressure Cal
Parameters} control (Fig.~\ref{fig:PresCal}). {\bf The parameters
need to be selected before scanning.} Except for Neg Peak
Pressure, they should only be used with a pulsed transducer. The
{\it Center Freq (MHz)} needs to be determined by examining the
spectrum of the signal in {\it SetUpDigi.vi}
(Sec.~\ref{sec:digi}). {\it PRF (kHz)} is the PRF setting on the
function generator. {\it Pulse Window ($\mu$s)} determines the
data window to compute the pulse parameters. Use {\it
SetUpDigi.vi} to see what time duration the pulse lasts. Finally,
select which Precision Acoustics Lab hydrophone is being used.

\begin{figure}[!htb]
\begin{center}
\includegraphics[width=1.5 in]{PresCal.eps}
 \caption{Input selections to obtain 510(k) parameters for transducer
 scan. }
 \label{fig:PresCal}
\end{center}
\end{figure}

The parameters related to the Pressure Cal Parameters are computed
from several parameters. The negative peak pressure $P_{min}$, a
derating factor $a$=0.3 dB/(cm MHz), and the pulse intensity
integral (PII)

\begin{equation}\label{eq:Pgas}
  PII=\int_{t_{0}}^{t_{1}}I(t) dt,
\end{equation}

where $t_{1}-t_{0}$ is {\it Pulse Window} and $I(t)$ is the
instantaneous pulse intensity ($I=p^{2}/\rho c$). From these
parameters we then find the pulse-average intensity
$I_{pa}=PII/(t_{1}-t_{0}$) and the temporal-average intensity
$I_{ta}=PII/PRF$. Derated values are then found by multiplying the
pressures by $a$ and the intensities by~$a^{2}$. The path length
used to compute a is assumed to be the separation between the
source and hydrophone. This may not relate to the physical
conditions in which the source is actually used.

\section{Viewing EYE files}

Use {\it ViewEyeFile.vi} to view an EYE file once it has been
saved to disc. This program will automatically create a sector
scan image if appropriate. A region of the EYE file can be viewed
as a 3-D intensity plot or as a single scan line can be displayed
as RF data. The region to view is selected with movable cursors in
the EYE file. The header info can also be viewed.

\subsection{Viewing a batch of EYE files as thumbnails}

To view a batch of saved EYE files as a collection of thumbnails
use {\it ViewPicts.vi}. When this program is executed, the user is
prompted to select a directory. Choose the directory, then select
the files to view in the list box display. When all the files are
selected, push {\it Use Selected Files}. After a short time the
thumbnails will be displayed with the file name overlaid on the
image. If the EYE file was a sector scan it will be converted to a
sector format.


\section{Altering Headers}

The VI {\it AlterHeader.vi} allows fields in a batch of EYE files
to be changed. The files that will be changed are selected like
what was discussed above in {\it ViewPicts.vi}. The fields that
get changed need to have a check in the box to their right and a
new value entered before pressing {\it Use Selected Files}. For
example, the case in figure \ref{fig:AltHead} has two fields
selected with check marks: \# rf lines per scan and sample rate
MHz. When {\it Use Selected Files} is pushed, the selected EYE
files (rri.eye for this example) will have their header updated
with the new values.


\begin{figure}[!htb]
\begin{center}
\includegraphics[width=5 in]{AlterHead.eps}
 \caption{Control panel of {\it AlterHeader.vi} which allows
 fields in the header of EYE files to be added or changed.}
 \label{fig:AltHead}
\end{center}
\end{figure}

\section{3D image prep}

There is now one application to prepare data for plotting in 3-D
({\it 3DVisPrep.vi}). The NoEsys software component T3D seems to
be the easiest application to view volumetric data. To read data
directly into T3D, it can be saved as a byte file or as ASCII
text. All data planes are saved in one file. The data can then be
read into T3D and the dimensions of ($X,Y,Z$) are entered. RGB
bitmap planes can also be converted to a single byte file.

The program asks for the root file name of all the planes, the
number of planes, and the directory of the files. For example, if
the data is in spreadsheet format and name {\it plane1.txt} and
{\it plane2.txt}, the root file name is plane and the number of
files is 2. {\bf Spreadsheet files need to end with {\it txt} and
RGB BMP files with {\it bmp}}. The user chooses what kind of
conversion to make and the output is saved to the chosen file
name. The output files are numbered if more than one is generated.




\end{document}
